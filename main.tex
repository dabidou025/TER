\documentclass[a4paper, 11pt, french]{article}
\usepackage[utf8]{inputenc}
\usepackage[margin=1.5cm,headheight=52pt,includeheadfoot]{geometry} 
\usepackage{mathtools} %amsmath mis à jour
\usepackage{amssymb} %symboles mathématiques
\usepackage{latexsym}
\usepackage{stmaryrd} 
\usepackage{babel} %gestion du français

\newcommand{\abs}[1]{\vert#1\vert}
\newcommand{\norme}[1]{\Vert#1\Vert}
\newcommand{\scal}[2]{<#1\vert#2>}


\title{TER M1 : \\ NN and RKHS}
\author{Matthieu Denis}

\begin{document}
	
	\maketitle
	\newpage
	
	\tableofcontents
	\newpage
	
	\section*{Introduction : Réseau de neurone simple}
	
	Commencons par étudier un NN très simple : une fonction
	$\Phi : (\mathbb{R}^m \times \mathbb{R}^{m \times m} \times \mathbb{R}^m) \times \mathbb{R} \to \mathbb{R}$ combinaison d'applications linéaires, sans non linéarités  intermédiaires  :
	
	\[ \Phi ((\beta, A, u), x) \coloneqq \frac{1}{m^{\alpha}} \beta^T
		 \left( \frac{1}{m^{\gamma}} A \right) u x \]
		 
	On initialise $\theta^0 \coloneqq (\beta^0, A^0, u^0)$ de manière standarde : 
	$ \forall i,j \in \{1, \cdots, m\} , \; u_i^0, A_{ij}^0, \beta_i^0 \sim_{iid} N(0, 1)$
	
	Nous montrerons quelques propriétés asymptotiques en la largeur des couches, et sur l'évolution des paramètres lors du premier pas de la descente de gradient.
	
	\subsection*{Lois suivant la largeur des couches $m$}
	
	\begin{itemize}
		
		\item Loi de $ || u^0 ||_2^2$ pour $m$ grand \\
		
		Comme $ || u^0 ||_2^2 = \sum_{i=1}^{m} (u_i^0)^2 \sim \chi^2 (m)$ et $ u_i^0 \sim \chi^2 (1)$, en appliquant le TCL aux $ u_i^0 $, on a :
		
		\[
			\frac{|| u^0 ||_2^2 - m}{\sqrt{2m}} \sim_{m \to \infty}  N(0, 1)
		\]
	
		C'est-à-dire que $ || u^0 ||_2^2 \sim N(m, 2m) $  pour $m$ grand. \\
		
		\item Loi de $ \left(\frac{1}{m^{\gamma}} A^0 \right) u^0 x $ sachant $ u^0 $ \\
		
		$ (A^0 u^0)_i = \sum_{j=1}^m A_{ij}^0 u_j^0$. En sachant $u^0$, comme 
		$A_{i \cdot}^0$ est un vecteur gaussien, $(A^0 u^0)_i \sim  N(0, || u^0 ||_2^2 ) $. 
	
		De même, part indépendance des $A_{ij}^0$, les $(A^0 u^0)_i$ sont indépendants et $A^0 u^0 \sim N(0_m, || u^0 ||_2^2 \; Id_m)$.
		
		Ainsi, $ \left(\frac{1}{m^{\gamma}} A^0 \right) u^0 x $ | $ u^0 
		\sim  N(0_m, \left( \frac{x}{m^{\gamma}} \right)^2 || u^0 ||_2^2 \; Id_m) $. \\
		
		\item Loi de $ \left(\frac{1}{m^{\gamma}} A^0 \right) u^0 x $ \\
		
		HISTOIRE DE MEME TRIBU ENGENDREE PAR U0 ET LE RESTE IMPLIQUE QUE CEST LA MEME CHOSE DECONDITIONNEE
		
		Donc $ \left(\frac{1}{m^{\gamma}} A^0 \right) u^0 x  \sim  N(0_m, 
		\left( \frac{x}{m^{\gamma}} \right)^2 || u^0 ||_2^2 \; Id_m) $. \\
		
		\item Loi de $ || \left(\frac{1}{m^{\gamma}} A^0 \right) u^0 x \; ||_2^2 $ pour $m$ grand
		 \\
		
		On a $ \left( \left(\frac{1}{m^{\gamma}} A^0 \right) u^0 x \right)_i^2 \sim  
		\left( \frac{x}{m^{\gamma}} \right)^2 || u^0 ||_2^2 \cdot  \chi^2 (1) $, d'espérance 
		$ \mu \coloneqq \left( \frac{x}{m^{\gamma}} \right)^2 || u^0 ||_2^2  $  et de variance \\
		$ \sigma^2 \coloneqq 2 \left( \left( \frac{x}{m^{\gamma}} \right)^2 || u^0 ||_2^2  \right)^2 $.
		
		Donc en appliquant le TCL à ceux ci, on a :
		
		 \[
		 	\frac{|| \left(\frac{1}{m^{\gamma}} A^0 \right) u^0 x \; ||_2^2 - m \mu}{\sigma \sqrt{m}} \sim_{m \to \infty}  N(0, 1)
		 \]
		
		C'est-à-dire que $|| \left(\frac{1}{m^{\gamma}} A^0 \right) u^0 x \; ||_2^2 \sim 
		N(m \mu, m \sigma^2) $  pour $m$ grand. \\
		
		\newpage
		
		\item Loi de $ \frac{1}{m^{\alpha}} (\beta^0)^T x_2 $ sachant $x_2$, avec 
		$x_2 = \left(\frac{1}{m^{\gamma}} A^0 \right) u^0 x$ \\
		
		On a 
		$ \frac{1}{m^{\alpha}} (\beta^0)^T x_2 | x_2 \sim N(0,  \frac{1}{m^{2\alpha}}||x_2||_2^2) $
		\\
		
		\item Loi de $ \frac{1}{m^{\alpha}} (\beta^0)^T x_2 $ \\
		
		On a 
		$ \frac{1}{m^{\alpha}} (\beta^0)^T x_2 \sim N(0,  \frac{1}{m^{2\alpha}}||x_2||_2^2) $
		\\
		
	\end{itemize}

	\subsection*{Choix de $\alpha$ et $\gamma$}
	
	On regarde la variance de chaque composante de 
	$ \left(\frac{1}{m^{\gamma}} A^0 \right) u^0 x $ pour $m$ grand : 
	$ Var = x^2 m^{-2\gamma} m $ comme $|| u^0 ||_2^2$ se comporte en $m$ pour $m$ grand. Or on ne veut pas qu'elle tende vers 0 ou l' infini lorsque $m$ tend vers l'infini car $\Phi$ prendrait des valeurs de 0 ou l'infini, ce qui impose le choix $\gamma = 1/2$ \\
	
	Quant au choix de $\alpha$, on a accès à la variance de  $ \frac{1}{m^{\alpha}} (\beta^0)^T x_2 $,
	qui est une v.a gaussienne pour $m$ grand. En prenant cette approximation, l'espérance de la variance de 
	$ \frac{1}{m^{\alpha}} (\beta^0)^T x_2 $ est $ m^{-2\alpha} m \mu $ pour $m$ grand, i.e 
	$  x^2 m^{-2\alpha} m \cdot m^{-2\gamma} m = x^2 m^{-2\alpha + 1} $ en prenant $\gamma = 1/2$. De la même manière que pour $\gamma$, on se retrouve avec le choix $\alpha = 1/2$ pour que la variance de $\Phi$ n'explose pas ni ne tende vers 0.
	
	\subsection*{Gradients}
	
	Trivialement,
	
	\[ \nabla_u \Phi = \frac{x}{m^{\alpha + \gamma}} \beta^T A \in \mathbb{R}^m\]
	
	\[ \nabla_{\beta} \Phi = \frac{x}{m^{\alpha + \gamma}} A u \in \mathbb{R}^m\]
	
	\[ \nabla_A \Phi = \frac{x}{m^{\alpha + \gamma}} \beta u^T \in \mathbb{R}^{m \times m}\]
	
	\subsection*{Descente de gradient}
	
	On va étudier ici le premier pas de descente de gradient.
	
	Posons une fonction de perte $F : \mathbb{R} \rightarrow \mathbb{R}$ t.q $F'(0) \neq 0$ et 
	$\Delta F \coloneqq F(\Phi(\theta^1, x)) - F(\Phi(\theta^0, x))$, avec 
	$\theta^1 \coloneqq \theta^0 - \eta \nabla_{\theta} F(\Phi(\theta^0, x))$

	Il semble honnête de prendre $\eta$ dépendant de $m$, le produit scalaire final ayant plus de chance d'exploser en grande dimension. Prenons $\eta \coloneqq m^a$, $a \in \mathbb{R}$ \\
	
	\begin{itemize}	
		\item Choix de $\eta$  \\
	\end{itemize}
		
	On veut que $\Delta F$ ne diverge pas ni ne tende vers 0 lorsque m tend vers l'infini.
	
	Pour cela, on utilise l'approximation 
	$\Delta F \simeq \; < \Delta \theta, \nabla_{\theta} F(\Phi(\theta^0, x)) >$.
	
	On a 
	\[
		\Delta F \simeq \; < -\eta \nabla_{\theta} F(\Phi(\theta^0, x)) , \nabla_{\theta} F(\Phi(\theta^0, x)) > = -\eta || \nabla_{\theta} F(\Phi(\theta^0, x)) ||^2
	\]
	
	\[
		\nabla_{\theta} F(\Phi(\theta^0, x)) = 
		\underbrace{F'(\Phi(\theta^0, x))}_\text{constante en $m$} 
		\cdot \nabla_{\theta} \Phi(\theta^0, x)
	\]
	
	Or $ || \nabla_{\theta} \Phi(\theta^0, x) ||^2 = || \nabla_{u} \Phi(\theta^0, x) ||^2
	+ || \nabla_{A} \Phi(\theta^0, x) ||^2 + || \nabla_{\beta} \Phi(\theta^0, x) ||^2 $ \\
	
	En faisant les mêmes types de calculs que dans la partie précédente, et en utilisant la loi des grands nombres, on trouve que pour $m$ grand :
	
	\[ || \nabla_{u} \Phi(\theta^0, x) ||^2 = C^2 \cdot || (\beta^0)^T A ||^2 \simeq
	cste \cdot C^2 \cdot m ||\beta^0||^2 \simeq cste \cdot C \cdot m^2 \; 
	\text{ avec $C \coloneqq \frac{x}{m^{\alpha + \gamma}}$} \]
	
	\[ || \nabla_{\beta} \Phi(\theta^0, x) ||^2 = C^2 \cdot || A u ||^2 \simeq
	cste \cdot C^2 \cdot m ||u^0||^2 \simeq cste \cdot C^2 \cdot m^2 \]
	
	Pour $|| \nabla_{A} \Phi(\theta^0, x) ||^2$, en considérant le gradient en $A$ comme un vecteur de la matrice des dérivées partielles applatie, on a 
	$|| \beta^0 (u^0)^T ||^2 = \sum_{i,j = 1}^m (\beta^0_i u^0_j)^2$. On fait face à une gaussiène puissance 4 : elle admet une espérance finie indépendante de $m$, donc en appliquant la LGN, on a :
	
	\[ || \nabla_{A} \Phi(\theta^0, x) ||^2 = C^2 \cdot || (\beta^0)^T A || \simeq
	cste \cdot C^2 \cdot m^2 \]
	
	Ainsi, pour $m$ grand :
	\begin{align*}
		\Delta F &\simeq -\eta || \nabla_{\theta} F(\Phi(\theta^0, x)) ||^2 \\
		&= - cste \cdot \eta || \nabla_{\theta} \Phi(\theta^0, x) ||^2 \\
		&\simeq - cste \cdot C^2 \eta (m^2 + m^2 + m^2) \\
		&= -cste \cdot m^{-2} m^a m^2 \; \text{ en prenant $\alpha = \gamma = 1/2$}\\
		&= -cste \cdot  m^a
	\end{align*}	
	
	Ce qui nous force le choix $a = 0$ pour que tout se passe bien. \\
	
	A présent, regardons les comment les paramètres ont évolué après ce premier pas de descente de gradient, i.e les ordres de grandeur des $\Delta$.
	On prend $\eta = \mathcal{O}(1)$, les résultats ci-dessus ne changent pas.

	
 \begin{align*}
	 	\Delta u &= - \eta \nabla_u F(\Phi(\theta^0, x)) \\
	 	&= - \eta F'(\Phi(\theta^0, x)) \cdot \nabla_{u} \Phi(\theta^0, x) \\
	 	&= -\eta \, C \, d \, (\beta^0)^T A^0 \; \text{avec $d \coloneqq F'(\Phi(\theta^0, x))$} \\
	 	&= -\eta \, x \, d \, m^{-1} \, (\beta^0)^T A^0
 \end{align*}

 	\begin{align*}
 		| (\Delta u)_i | &= \eta \, |x \, d| \, m^{-1} 
 		\left| \sum_{j = 1}^{m} A^0_{ji} \beta^0_j \right| \\
 		&\leq \eta \, |x \, d| \, m^{-1} \sum_{j = 1}^{m} | A^0_{ji} \beta^0_j | \\
 		&\leq  \eta \, |x \, d| \, m^{-1} \cdot m \sup_j | A^0_{ji} \beta^0_j | \\
 		&\leq \eta \, |x \, d| \, \sup_{ij} | A^0_{ji} \beta^0_j |
 	\end{align*}
	
		Donc
		
		\begin{align*}
			|| \Delta u|| &= \left( \sum_{i = 1}^{m} | (\Delta u)_i |^2 \right)^{1/2} \\
			&\leq ( m \, (\eta \, |x \, d| \, \sup_{ij} | A^0_{ji} \beta^0_j | )^2 )^{1/2} \\
			&= \eta \, |x \, d| \, \sup_{ij} | A^0_{ji} \beta^0_j | \cdot m^{1/2} \\
			&= \mathcal{O}(m^{1/2})
		\end{align*}
	
	De la même manière :
	
	\begin{align*}
			||\Delta \beta|| &\leq \eta \, |x \, d| \, \sup_{ij} | A^0_{ij} u^0_j | \cdot m^{1/2} \\
		&= \mathcal{O}(m^{1/2})
	\end{align*}

	\begin{align*}
		|(\Delta A)_{ij}| &= | -\eta \, x \, d \, m^{-1} \, (\beta^0 (u^0)^T)_{ij} | \\
		&= \eta \, |x \, d| \, m^{-1} \, |\beta^0_i u^0_j| \\
		&\leq \eta \, |x \, d| \, m^{-1} \, \sup_{ij} |\beta^0_i u^0_j| \\
		&= \mathcal{O}(m^{-1})
	\end{align*}

	Donc
	
	\begin{align*}
		||\Delta A||_F &= \left( \sum_{i,j = 1}^{m} | (\Delta A)_{ij} |^2 \right)^{1/2} \\
		&\leq ( m^2 (\eta \, |x \, d| \, m^{-1} \, \sup_{ij} |\beta^0_i u^0_j|)^2 )^{1/2} \\
		&= \eta \, |x \, d| \, \sup_{ij} |\beta^0_i u^0_j| \\
		&= \mathcal{O}(1)
	\end{align*}

	On peut faire mieux pour $||\Delta u||$ et $||\Delta \beta||$, en utilisant leurs propriétés aléatoires :
	\begin{align*}
		||\Delta u|| &=  \eta \nabla_u F(\Phi(\theta^0, x)) \\
		&= \eta \, |x \, d| \, m^{-1} \, ||(\beta^0)^T A^0||
	\end{align*}
	
	Or $||(\beta^0)^T A^0||^2 \simeq ||\beta^0||m$ et $||\beta^0|| \simeq m$ pour m grand, donc $||(\beta^0)^T A^0|| \simeq m$ pour m grand.\\
	
	D'où $ ||\Delta u|| = \mathcal{O}(1) $ pour m grand. Et par la même gymnastique, $ ||\Delta \beta|| = \mathcal{O}(1) $ pour m grand.
	
	\newpage
	
	\subsection{$\gamma = 1/2$}
	
	\[\nabla_u \Phi = \frac{x}{m^{\alpha + 1/2}} A^T \beta\]
	
\end{document}