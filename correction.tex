\documentclass[a4paper, 11pt, french]{article}
\usepackage[utf8]{inputenc}
\usepackage[margin=1.5cm,headheight=52pt,includeheadfoot]{geometry} 
\usepackage{mathtools} %amsmath mis à jour
\usepackage{amssymb} %symboles mathématiques
\usepackage{latexsym}
\usepackage{stmaryrd} 
\usepackage{babel, ulem, xcolor} %gestion du français

\renewcommand{\labelitemi}{∙}

\newcommand{\abs}[1]{\vert#1\vert}
\newcommand{\norme}[1]{\Vert#1\Vert}
\newcommand{\scal}[2]{<#1\vert#2>}
\newcommand{\ch}[1]{{\color{blue} [ #1 ]}}


\title{TER M1 : \\ NN and RKHS}
\author{Matthieu Denis}

\begin{document}
	
	\maketitle
	\newpage
	
	\tableofcontents
	\newpage
	
	\section{Introduction : Réseau de neurone simple}
	
	Commencons par étudier un NN très simple : une fonction
	$\Phi : (\mathbb{R}^m \times \mathbb{R}^{m \times m} \times \mathbb{R}^m) \times \mathbb{R} \to \mathbb{R}$ combinaison d'applications linéaires, sans non linéarités  intermédiaires  :
	
	\[ \Phi ((\beta, A, u), x) \coloneqq \frac{1}{m^{\alpha}} \beta^T
		 \left( \frac{1}{m^{\gamma}} A \right) u x \]
		 
	On initialise $\theta^0 \coloneqq (\beta^0, A^0, u^0)$ de manière standarde : 
	$ \forall i,j \in \{1, \cdots, m\} , \; u_i^0, A_{ij}^0, \beta_i^0 \sim_{iid} N(0, 1)$
	
	Nous montrerons quelques propriétés asymptotiques en la largeur des couches, et sur l'évolution des paramètres lors du premier pas de la descente de gradient.
	
	\subsection{Lois suivant la largeur des couches $m$}
	
	\begin{itemize}
		
		\item[$\bullet$] Loi de $ || u^0 ||_2^2$ pour $m$ grand \\
		
		Comme $ || u^0 ||_2^2 = \sum_{i=1}^{m} (u_i^0)^2 \sim \chi^2 (m)$ et $ u_i^0 \sim \chi^2 (1)$, en appliquant le TCL aux $ u_i^0 $, on a :
		
		\[
			\frac{|| u^0 ||_2^2 - m}{\sqrt{2m}} \sim_{m \to \infty}  N(0, 1)
		\]
	
		\sout{C'est-à-dire que $ || u^0 ||_2^2 \sim N(m, 2m) $} \ch{en particulier, $ || u^0 ||_2^2 \sim m $}  pour $m$ grand. \\
		
		\item[$\bullet$] Loi de $ \left(\frac{1}{m^{\gamma}} A^0 \right) u^0 x $ sachant $ u^0 $ \\
		
		$ (A^0 u^0)_i = \sum_{j=1}^m A_{ij}^0 u_j^0$. En sachant $u^0$, comme 
		$A_{i \cdot}^0$ est un vecteur gaussien, $(A^0 u^0)_i \sim  N(0, || u^0 ||_2^2 ) $. 
	
		De même, part indépendance des $A_{ij}^0$, les $(A^0 u^0)_i$ sont indépendants et \ch{conditionnellement \`a $u_{0}$} $A^0 u^0 \sim N(0_m, || u^0 ||_2^2 \; Id_m)$.
		
		Ainsi, \ch{la loi de } $ \left(\frac{1}{m^{\gamma}} A^0 \right) u^0 x $ \ch{sachant} $ u^0$ et 
		$N(0_m, \left( \frac{x}{m^{\gamma}} \right)^2 || u^0 ||_2^2 \; Id_m) $. \\
		
		\ch{Voici l'argumentation pour la suite\\
		Lemme: si $X_{n} \sim N(\mu_{n},\sigma_{n})$ avec $\mu_{n}\to\mu$ et $\sigma_{n}\to \sigma$, alors $(X_{n})$ converge en loi vers $X_{\infty} \sim N(\mu,\sigma)$
		
		cons\'equence: si $\gamma < 1/2$ la loi conditionnelle de $\left(\frac{1}{m^{\gamma}} A^0 u^0 x\right)_{i} $ sachant $u^0$ converge vers une dirac en zero, donc $\left(\frac{1}{m^{\gamma}} A^0 u^0 x\right)_{i} $ converge en loi (et en proba) vers 0.
		Si $\gamma=1/2$ les lois de $\left(\frac{1}{m^{\gamma}} A^0  u^0 x\right)_{i}$ sachant $u^0$ converge vers une $N(0,x^2)$. Comme cette loi est ind\'ependante de $u^0$, on a donc que $\left(\frac{1}{m^{\gamma}} A^0  u^0 x\right)_{i}$ converge en loi vers une variable de loi $N(0,x^2)$.
		Si $\gamma > 1/2$, $\left(\frac{1}{m^{\gamma}} A^0 u^0 x \right)_{i}$ diverge vers l'infini.}
		
		
		\item[$\bullet$] Loi de $ \left(\frac{1}{m^{\gamma}} A^0 \right) u^0 x $ \\
		
		HISTOIRE DE MEME TRIBU ENGENDREE PAR U0 ET LE RESTE IMPLIQUE QUE CEST LA MEME CHOSE DECONDITIONNEE
		
		Donc $ \left(\frac{1}{m^{\gamma}} A^0 \right) u^0 x  \sim  N(0_m, 
		\left( \frac{x}{m^{\gamma}} \right)^2 || u^0 ||_2^2 \; Id_m) $. \\
		
		\item[$\bullet$] Loi de $ || \left(\frac{1}{m^{\gamma}} A^0 \right) u^0 x \; ||_2^2 $ pour $m$ grand
		 \\
		
		On a $ \left( \left(\frac{1}{m^{\gamma}} A^0 \right) u^0 x \right)_i^2 \sim  
		\left( \frac{x}{m^{\gamma}} \right)^2 || u^0 ||_2^2 \cdot  \chi^2 (1) $, d'espérance 
		$ \mu \coloneqq \left( \frac{x}{m^{\gamma}} \right)^2 || u^0 ||_2^2  $  et de variance \\
		$ \sigma^2 \coloneqq 2 \left( \left( \frac{x}{m^{\gamma}} \right)^2 || u^0 ||_2^2  \right)^2 $.
		
		Donc en appliquant le TCL à ceux ci, on a :
		
		 \[
		 	\frac{|| \left(\frac{1}{m^{\gamma}} A^0 \right) u^0 x \; ||_2^2 - m \mu}{\sigma \sqrt{m}} \sim_{m \to \infty}  N(0, 1)
		 \]
		
		C'est-à-dire que $|| \left(\frac{1}{m^{\gamma}} A^0 \right) u^0 x \; ||_2^2 \sim 
		N(m \mu, m \sigma^2) $  pour $m$ grand. \\
		
		\newpage
		
		\item[$\bullet$][$\bullet$] Loi de $ \frac{1}{m^{\alpha}} (\beta^0)^T x_2 $ sachant $x_2$, avec 
		$x_2 = \left(\frac{1}{m^{\gamma}} A^0 \right) u^0 x$ \\
		
		On a 
		$ \frac{1}{m^{\alpha}} (\beta^0)^T x_2 | x_2 \sim N(0,  \frac{1}{m^{2\alpha}}||x_2||_2^2) $
		\\
		
		\item[$\bullet$][$\bullet$] Loi de $ \frac{1}{m^{\alpha}} (\beta^0)^T x_2 $ \\
		
		On a 
		$ \frac{1}{m^{\alpha}} (\beta^0)^T x_2 \sim N(0,  \frac{1}{m^{2\alpha}}||x_2||_2^2) $
		\\
		
	\end{itemize}

	\subsection{Choix de $\gamma$}
	
	On regarde la variance de chaque composante de 
	$ \left(\frac{1}{m^{\gamma}} A^0 \right) u^0 x $ pour $m$ grand : 
	$ Var = x^2 m^{-2\gamma} m $ comme $|| u^0 ||_2^2$ se comporte en $m$ pour $m$ grand. Or on ne veut pas qu'elle tende vers 0 ou l' infini lorsque $m$ tend vers l'infini car $\Phi$ prendrait des valeurs de 0 ou l'infini, ce qui impose le choix $\gamma = 1/2$ \\
	
	Dans la suite, on prendra $\gamma = 1/2$.

	\subsection{Gradients}
	
	Trivialement,
	
	\[\nabla_{\beta} \Phi = \frac{x}{m^{\alpha + 1/2}} A u\]
	
	\[\nabla_u \Phi = \frac{x}{m^{\alpha + 1/2}} A^T \beta\]

	\[\nabla_A \Phi = \frac{x}{m^{\alpha + 1/2}} \beta u^T\]
	
	Etudions les ordres de grandeur des normes correspondantes à l'initialisation pour $m$ grand :
	
	On va simplement utiliser les approximations données par la loi des grands nombres :
	 $||u^0|| \simeq \sqrt{m}$, et comme vu plus haut, 
	 $||A^0 u^0|| \simeq \sqrt{m}||u^0|| \simeq m$. Ainsi $||\nabla_{\beta} \Phi (\theta^0, x)|| \sim m^{-\alpha - 1/2} \cdot m = m^{1/2 - \alpha}$.
	 
	 \[||\nabla_{\beta} \Phi (\theta^0, x)|| \sim m^{1/2 - \alpha}\]
	 
	 Exactement de la même manière, on aboutit à :
	 
	 \[||\nabla_u \Phi (\theta^0, x)|| \sim m^{1/2 - \alpha}\]
	 
	 Pour $A$, on prend la norme de Frobenius : la LGN nous dit que $||\beta^0 (u^0)^T|| \simeq m$ \ch{\`a d\'etailler} et donc :
	 
	 \[||\nabla_A \Phi (\theta^0, x)||_F \sim m^{1/2 - \alpha}\]
	 
	 \subsection{Descente de gradient}
	 
	 On va étudier ici le premier pas de descente de gradient.
	 
	 Posons une fonction de perte $F : \mathbb{R} \rightarrow \mathbb{R}$ t.q $F'(0) \neq 0$ et 
	 $\Delta F \coloneqq F(\Phi(\theta^1, x)) - F(\Phi(\theta^0, x))$, avec 
	 $\theta^1 \coloneqq \theta^0 - \eta \nabla_{\theta} F(\Phi(\theta^0, x))$
	 
	 Il semble honnête de prendre $\eta$ dépendant de $m$, le produit scalaire final ayant plus de chance d'exploser en grande dimension. Prenons $\eta \coloneqq m^a$, $a \in \mathbb{R}$ \\
	 
	 \subsubsection{Choix de $\eta$}
	 
	 On veut que $\Delta F$ ne diverge pas ni ne tende vers 0 lorsque m tend vers l'infini.
	 
	 Pour cela, on utilise l'approximation 
	 $\Delta F \simeq \; < \Delta \theta, \nabla_{\theta} F(\Phi(\theta^0, x)) >$.
	 
	 On a 
	 \[
	 \Delta F \simeq \; < -\eta \nabla_{\theta} F(\Phi(\theta^0, x)) , \nabla_{\theta} F(\Phi(\theta^0, x)) > = -\eta || \nabla_{\theta} F(\Phi(\theta^0, x)) ||^2
	 \]
	 
	 \[
	 \nabla_{\theta} F(\Phi(\theta^0, x)) = 
	 \underbrace{F'(\Phi(\theta^0, x))}_\text{constante en $m$} 
	 \cdot \nabla_{\theta} \Phi(\theta^0, x)
	 \]
	 
	 Or $ || \nabla_{\theta} \Phi(\theta^0, x) ||^2 = || \nabla_{\beta} \Phi(\theta^0, x) ||^2 + || \nabla_{u} \Phi(\theta^0, x) ||^2 + || \nabla_{A} \Phi(\theta^0, x) ||^2$ \\
	 
	 Donc pour $m$ grand :
	 \begin{align*}
	 	\Delta F &\simeq -\eta || \nabla_{\theta} F(\Phi(\theta^0, x)) ||^2 \\
	 	&\sim \eta || \nabla_{\theta} \Phi(\theta^0, x) ||^2 \\
	 	&\simeq \eta (3 \cdot (m^{1/2 - \alpha})^2) \\
	 	&\simeq m^a \cdot m^{1 - 2\alpha}
	 \end{align*}	
	
	Ce qui impose le choix $a = 2\alpha - 1$ \\
	
	\subsubsection{Ordres de grandeur des écarts relatifs}

	Pour cela introduisons $\Delta \theta \coloneqq \theta^1 - \theta^0$. Remarquons que $	\Delta \theta = - \eta \nabla_{\theta} F(\Phi(\theta^0, x))$.
	
	\begin{align}
		||\Delta u|| &\sim \eta || \nabla_u \Phi(\theta^0, x) || \\
		&\sim m^{2\alpha - 1} \cdot m^{1/2 - \alpha} \\
		&\sim m^{\alpha - 1/2}
	\end{align}

	Ce qui nous donne un ordre de grandeur de l'écart relatif :
	
	\[\frac{||\Delta u||}{||u^0||} \sim m^{\alpha - 1}\]
	
	On a le même résultat pour l'écart relatif de $\beta^0$ :
	
	\[\frac{||\Delta \beta||}{||\beta^0||} \sim m^{\alpha - 1}\]
	
	Pour $A$, la LGN nous donne $||A^0||_F \simeq m$ pour $m$ grand, on a alors par les mêmes calculs :
	
	\[\frac{||\Delta A||_F}{||A^0||_F} \sim m^{\alpha - 3/2}\]
	
	Concernant l'écart entrywise de A, on a $|\Delta A_{ij}| \sim \eta | (\nabla_A \Phi(\theta^0, x))_{ij} | \sim m^{2\alpha - 1} \cdot m^{-1/2 - \alpha} \cdot 1 \sim m^{\alpha - 3/2}$ car $|\beta^0 (u^0)^T| \sim 1$. $|A_{ij}| \sim 1$, donc :

	\[\frac{|\Delta A_{ij}|}{|A_{ij}|} \sim m^{\alpha - 3/2}\]
	
	Maintenant avec la norme opétateur : le corollaire 7.9 du cours de MIA2 nous donne une majoration sur $|A^0|_{op}$ : $|A^0|_{op} \leq \sqrt{m} + 7\sqrt{m + \xi} \sim \sqrt{m}$, \ch{plut\^ot $O(\sqrt{m})$ ici} avec $\xi \sim Exp(1)$.
	
	De plus, on peut trouver la la norme opérateur de $\Delta A$ comme suit : tout le travail est de trouver $|\beta^0 (u^0)^T|_{op} \coloneqq \sup \{||\beta^0 (u^0)^T x|| \; 
	\text{avec }||x|| = 1\}$.
	
	\begin{align}
		(\beta^0 (u^0)^T x)_{ij} &= \beta^0_i u^0_j\ \ch{soucis\ dans\ cette\ formule}\\
		(\beta^0 (u^0)^T x)_i &= \sum_{j=1}^{m} (\beta^0 (u^0)^T x)_{ij} \cdot x_j\ \ch{ici\ aussi} \\
		&= \beta^0_i < u^0, x> \\
		||\beta^0 (u^0)^T x|| &= |< u^0, x>| \cdot ||\beta^0||
	\end{align}

	Donc le $\sup$ est bien atteint en $x = u / ||u^0||$ et est égal à $||u^0|| \cdot ||\beta^0||$. En utilisant les approximations précédentes, on a donc $|\beta^0 (u^0)^T|_{op} \simeq m$. Ainsi on a $|\Delta A|_{op} \sim m^{2\alpha - 1} \cdot m^{-1/2 - \alpha} \cdot m \sim m^{\alpha - 1/2}.$ et :
	
	\[\frac{|\Delta A|_{op}}{|A^0|_{op}} \sim m^{\alpha - 1}\]
	
	\subsubsection{Choix de $\alpha$}
	
	\begin{itemize}
		\item[$\bullet$] $\alpha < 1$ \\
		
		Dans ce cas là, tous les écarts relatifs d'ordre $m^{\alpha - 1} \xrightarrow[m \to \infty]{} 0$. On a donc pour $m$ grand :
		
		\[||\Delta u|| \ll ||u^0||\]
		
		\[||\Delta \beta|| \ll ||\beta^0||\]
		
		\[|\Delta A|_{op} \ll |A^0|_{op}\]

		Regardons maintenant les $\Delta$ des gradients en $u$ pour la première itération :
		
		\begin{align}
			\Delta \nabla_u \Phi &\coloneqq \nabla_u \Phi (\theta^1, x) -  \nabla_u \Phi (\theta^0, x) \\
			&\sim m^{-\alpha - 1/2} \underbrace{((\beta^1)^T A^1 - (\beta^0)^T A^0)}_\text{($\star$)} \\
			 (\star) &= (\beta^0 - \eta \nabla_{\beta} F(\Phi(\theta^0, x)))^T (A^0 - \eta \nabla_{A} F(\Phi(\theta^0, x))) - (\beta^0)^T A^0 \\
			&= \eta^2 [\nabla_{\beta} F(\Phi)]^T[\nabla_{A} F(\Phi)] - \eta [\nabla_{\beta} F(\Phi)] A^0 - \eta \beta^0 [\nabla_{A} F(\Phi)] \\
			&= (\Delta \beta)^T(\Delta A) - (\Delta \beta) A^0 -  \beta^0 (\Delta A)\ \ch{des \ +\ partout\ non?}
		\end{align}
	
	Donc :
	
	\begin{align}
		||\Delta \nabla_u \Phi|| &\sim m^{-\alpha - 1/2} ||(\Delta \beta)^T(\Delta A) - (\Delta \beta) A^0 -  \beta^0 (\Delta A)|| \\
		&\lesssim m^{-\alpha - 1/2} (||\Delta \beta|| \cdot |\Delta A|_{op} + ||\Delta \beta|| \cdot |A^0|_{op} + ||\beta^0|| \cdot |\Delta A|_{op}) \\
		&\lesssim m^{-\alpha - 1/2} (m^{\alpha - 1/2}m^{\alpha - 1/2} + m^{\alpha - 1/2}m^{1/2} + m^{1/2}m^{\alpha - 1/2}) \\
		&\lesssim m^{-\alpha - 1/2} (m^{\alpha - 1/2}m^{1/2} + m^{\alpha - 1/2}m^{1/2} + m^{1/2}m^{\alpha - 1/2}) \\
		&\lesssim m^{-1/2}
	\end{align}

	C'est-à-dire que $||\Delta \nabla_u \Phi|| = \mathcal{O}(m^{-1/2})$ pour $m$ grand. \\
	
	Par la même démarche on trouve $||\Delta \nabla_{\beta} \Phi|| = \mathcal{O}(m^{-1/2})$ et $||\Delta \nabla_{A} \Phi||_F = \mathcal{O}(m^{-1/2})$ pour $m$ grand. \\
	
	Ainsi : \ch{c'est toujours un peu compliqu\'e de discuter les ordres de grandeurs sans \'ecrire des choses inexactes. En maths $a_{m}\sim b_{m}$ signifie que $a_{m}/b_{m}\to 1$. Pour dire qu'on a simultan\'ement $a_{m}=O(b_{m})$ et $b_{m}=O(a_{m})$ on note souvent (m\^eme si \c{c}a n'est pas un standard universel) $a_{m}\asymp b_{m}$. C'est une notation que tu peux utilement introduire en d\'ebut de manuscrit, elle te sera utile pour discuter les ordres de grandeur}
	
	\[||\Delta \nabla_{\beta / u / A} \Phi|| = \mathcal{O}(m^{-1/2}) \stackrel{\alpha < 1}{\ll} m^{1/2 - \alpha} \sim ||\nabla_{\beta / u / A} \Phi||\]
	
	Cela traduit un comportement linéaire lorsque $\alpha < 1$ pour $m$ grand. \\
	
	On peut aussi remarquer que
	
	\[\Delta \nabla_{\theta} F(\Phi) =  F'(\Phi) \cdot \Delta \nabla_{\theta} \Phi \]
	
	On a aussi :

	\[||\Delta \nabla_{\beta / u / A} F(\Phi) || = \mathcal{O}(m^{-1/2}) \stackrel{\alpha < 1}{\ll} m^{1/2 - \alpha} \sim ||\nabla_{\beta / u / A} F(\Phi)||\]

	\ch{ce qu'il faut discuter, c'est la chose suivante: on voit qu'apr\`es une \'etape de descente de gradient, $\Delta F\asymp 1$ mais les variations relatives des param\`etres et gradients tendent vers 0. Donc \`a l'\'etape suivante, les \'evaluations pr\'ec\'edentes restent valables et le m\^eme ph\'enom\`ene se reproduira. In fine, apr\`es un nombre fini d'\'etape de gradient, on aura toujours $F(\Phi(\theta^t, x)) = F(\Phi(\theta^0, x)) + \langle \theta^t-\theta^0 , \nabla_{\theta} F(\Phi(\theta^0, x))  \rangle  +O(m^{-1/2})$. Cela signifie donc qu'on apprend un mod\`ele lin\'eaire relatif aux features $\nabla_{\theta} F(\Phi(\theta^0, x))$. Autrement dit on fait un apprentissage dans un RKHS de noyau
	$$k(x,y)=\langle\nabla_{\theta} F(\Phi(\theta^0, x)),\nabla_{\theta} F(\Phi(\theta^0, y))\rangle \stackrel{LGN}{=} \mathbb{E}[\langle\nabla_{\theta} F(\Phi(\theta^0, x)),\nabla_{\theta} F(\Phi(\theta^0, y))\rangle].$$
	En particulier, le noyau ne d\'epend que de l'architecture du r\'eseau de neurones et de l'initialisation, pas des donn\'ees (no feature learning).
	
	Tu peux aussi mettre a remarque que pour $\alpha=1$ cela n'est plus vrai, il y a bien feature learning.
	
	Ici on a fait le calcul dans le cas le plus simple possible pour voir apparaitre le ph\'enom\`ene. Dans la suite on va: voir que le calcul reste valable pour $\sigma(x)\neq x$ et $x\in R^d$. Puis on d\'erivera proprement le r\'esultat. 
	}
		
	\end{itemize}

	
	
\end{document}